\begin{frame}

\begin{center}

\large \textbf{A ``container'' is a process that wraps another process}

\end{center}

\pause

\vspace{\fill}

Use cases:

\begin{itemize}

\item Set up a temporary resource, and tear it down subsequently.

\begin{itemize}

\item Temporary directories, temporarily pivot the root, etc.

\end{itemize}

\item Run the subprocess with reduced priveleges.

\begin{itemize}

\item Principle of Least Privelege

\end{itemize}

\item Monitor the subprocess.

\begin{itemize}

\item \texttt{strace}, \texttt{time}, etc.

\end{itemize}

\end{itemize}

\end{frame}


\begin{frame}

\frametitle{Containers: Benefits}

Piggyback on the operating system process abstraction:

\begin{itemize}

\item Robustness

\begin{itemize}

\item Control returns to the parent process regardless of whether the process
exitted normally or e.g. segfaulted.

\end{itemize}

\item Isolation

\begin{itemize}

\item Rely on the underlying operating system to enforce, memory, filesystem,
and other resource isolations.

\end{itemize}

\end{itemize}

\end{frame}
