\begin{frame}

\question{What is the best way to create a temporary directory, a
``scratchspace''?}

\pause

\answer{Use a container!}

\begin{itemize}

\item Create a temporary directory (TMPDIR).

\item Pass TMPDIR as a parameter to a subprocess.

\item Destroy TMPDIR when subprocess exits.

\end{itemize}

\end{frame}


\begin{frame}

\begin{center}

\Huge \textbf{DEMO}

\bigskip

\large \texttt{tmpdir}

\end{center}

\end{frame}


\begin{frame}[fragile]

\frametitle{\texttt{tmpdir} --- Robust and Versatile}

\begin{center}

Many programming languages come with ``dispose patterns'':

\end{center}

\begin{columns}

\begin{column}{0.3\textwidth}

\begin{lstlisting}[language=java]
try {
  ...
}
finally {
  ...
}
\end{lstlisting}

\end{column}

\begin{column}{0.3\textwidth}

\begin{lstlisting}[language=java]
using (... = ...)
{
  ...
}
\end{lstlisting}

\end{column}

\begin{column}{0.3\textwidth}

\begin{lstlisting}[language=python]
with ... as ...:
  ...
\end{lstlisting}

\end{column}

\end{columns}

\begin{center}

But they don't (trivially) work for \underline{\textbf{arbitrary programs}}.

\end{center}

\end{frame}
